\documentclass[10pt,a4paper]{article}
\usepackage[utf8]{inputenc}
\usepackage[spanish]{babel}
\usepackage{amsmath}
\usepackage{amsfonts}
\usepackage{amssymb}
\usepackage{graphics}
\usepackage{graphicx}
\usepackage{xcolor}
\usepackage{listings}
\usepackage{csvsimple}
\usepackage{caption}
\usepackage{subcaption}
\usepackage{enumitem}
\usepackage[left=2cm,right=2cm,top=2cm,bottom=2cm]{geometry}

\renewcommand*\contentsname{Índice} %Nombre del indice

\definecolor{codegreen}{rgb}{0,0.6,0}
\definecolor{codegray}{rgb}{0.5,0.5,0.5}
\definecolor{codepurple}{rgb}{0.58,0,0.82}
\definecolor{backcolour}{rgb}{0.95,0.95,0.92}

\lstdefinestyle{mystyle}{
    backgroundcolor=\color{backcolour},   
    commentstyle=\color{codegreen},
    keywordstyle=\color{magenta},
    numberstyle=\tiny\color{codegray},
    stringstyle=\color{codepurple},
    basicstyle=\ttfamily\footnotesize,
    breakatwhitespace=false,         
    breaklines=true,                 
    captionpos=b,                    
    keepspaces=true,                 
    numbers=left,                    
    numbersep=5pt,                  
    showspaces=false,                
    showstringspaces=false,
    showtabs=false,                  
    tabsize=2
}

\lstset{style=mystyle}

\begin{document}
\lstset{
	basicstyle=\footnotesize,
	extendedchars=true,
	literate={á}{{\'a}}1 {ã}{{\~a}}1 {é}{{\'e}}1 {ú}{{\'u}}1 {ó}{{\'o}}1,
	backgroundcolor=\color{black!5}
	}
	
\begin{titlepage}
	\centering
	{\includegraphics[scale=0.5]{Logo_UGR.png}\par}
	\vspace{1cm}
	{\bfseries\Large Facultad de Ciencas \par}
	\vspace{2.5cm}
	{\scshape\Huge Entrega de ejercicios 2: Bases Estándar\par}
	\vspace{3cm}
	{\itshape\Large Doble Grado Ingeniería Informática y Matemáticas}
	\vfill
	{\Large Autores: \par}
	{\Large Javier Gómez López \par}
	{\Large Pablo Fuentes Jimenez \par}
	{\Large Alejandro Rubio Martínez \par}
	{\Large Inmaculada García Moreno \par}
	{\Large Juan Valentín Guerrero Cano \par}
	
	\vfill
	{\Large 23 de noviembre de 2023 \par}
\end{titlepage}

\thispagestyle{empty}
\null
\vfill

%%Información sobre la licencia
\parbox[t]{\textwidth}{
  \includegraphics[scale=0.05]{by-nc-sa.png}\\[4pt]
  \raggedright % Texto alineado a la izquierda
  \sffamily\large
  {\Large Este trabajo se distribuye bajo una licencia CC BY-NC-SA 4.0.}\\[4pt]
  Eres libre de distribuir y adaptar el material siempre que reconozcas a los\\
  autores originales del documento, no lo utilices para fines comerciales\\
  y lo distribuyas bajo la misma licencia.\\[4pt]
  \texttt{creativecommons.org/licenses/by-nc-sa/4.0/}
}

\newpage

Sea \(R = \mathbb{C}[x,y]\) con el orden monomial \(>_{dp}\). Tomamos el polinomio \(f(x,y) = y^5 - x^7 + ax^3 y^3 + b x^4 y^4\) con \(a,b \in \mathbb{C}\). Se denomina ideal de Tjurina \((J)_f\) de \(f\) al ideal
\[
    \mathcal{J}_f := \left< f,  \partial f / \partial x, \partial f / \partial y\right>
\]

\begin{enumerate}[label=(\alph*)]
	\item Usando el algoritmo 1, calcula las posibles bases estándar de \(\mathcal{J}_f\) en función del valor de los parámetros \(a\) y \(b\).
	
	En nuestro caso, hemos desarrollado en algoritmo en \texttt{Python}, y posteriormente hemos usado \texttt{Singular} para simplificar los resultados obtenidos y comprobarlos. 
	
	Nuestro programa de \texttt{Python} es el siguiente:
	
	\lstinputlisting[language=Python]{../obtener_base.py}
	
	Vamos a explicar cada función del código:
	\begin{itemize}
		\item \texttt{spoly}: Función que calcula el S-polinomio tal y cómo se explica en la referencia del libro.
		\item \texttt{NF\_Butcher}: función que calcula la forma normal de Butcher según la referencia del libro. Será el algoritmo de forma normal que usaremos para obtener nuestra base estándar.
		\item \texttt{get\_standart\_base}: el algoritmo que se encuentra al principio de la hoja implementado en \texttt{Python}.
		\item \texttt{Tjurina\_generator}: función que calcula el ideal de Tjurina dado un polinomio.
	\end{itemize}
	
	Para el estudio de las posibles bases estándar, vamos a distinguir casos:
	\begin{itemize}
		\item \( a=b=0\).
		
		En este caso, el resultado que nos arroja \texttt{Python} es el siguiente:
			\[
				\{ -x^7 + y^5, -7 x^6, 5 y^4 \}
			\]
			
			Lo que vamos a hacer ahora es usar \texttt{Singular} para simplificar nuestra base. Para ello, realizamos los siguientes comandos:
			\begin{lstlisting}[language=Python]
				ring R = (0,a,b),(x,y),dp; //Definimos nuestra localizacion
				ideal S = -x^7 + y^5, -7*x^6, 5*y^4;
				S = simplify(S,32); S = simplify(S,2); S;
			\end{lstlisting}
			Y nos da la siguiente salida:
			\[
				\{-7x^6, 5y^4\}
			\]
			El comando \texttt{simplify} simplifica nuestro ideal. El argumento 32 borra los generadores que sus términos líderes son los por los términos líderes de otros generadores. El argumento 2 borra los generadores que sean 0. Para comprobar que el resultado es correcto, usaremos la función \texttt{std} de \texttt{Singular}, que calcula bases estándar:
			\begin{lstlisting}[language=Python]
				poly f = y^5 - x^7;
				ideal S = f, jacob(f);
				std(S);
			\end{lstlisting}
			Y nos da el siguiente resultado:
			\[
				\{x^6, y^4\}
			\]
			que es el mismo resultado que nuestra base simplificando (salvo constantes). En los casos que siguen, el método que usamos es el mismo.
			
			\medskip
			
			\item \(a = 0, b \neq 0\)
			
			En este caso, el resultado que nos arroja \texttt{Python} es el siguiente:
			\[
			\{bx^4y^4 - x^7 + y^5,4bx^3y^4 - 7x^6,4bx^4y^3 + 5y^4,3x^7 + 4y^5,13y^5,-195x^3y^4,91x^6y,-1365x^6,-455x^2y^4\}
			\]
			
			Una vez simplificado en \texttt{Singular}, obtenemos que
			\[
				\{13 y^5, -455 x^2 y^4, -1365 x^6, 4b x^4 y^3 + 5 y^4\}
			\]
			
			La comprobación de \texttt{Singular} nos dice la base estándar correcta es:
			\[
				\{y^5, x^2 y^4, x^6, 4b x^4y^3 + 5 y^4\}
			\]
			luego, nuestra solución es correcta (salvo constante).

			\medskip			
			
			\item \(a \neq 0, b = 0\)
			
			En este caso, el resultado que nos arroja \texttt{Python} es el siguiente:
			\[
			\{ax^3y^3 - x^7 + y^5,3ax^2y^3 - 7x^6,3ax^3y^2 + 5y^4,-1y^5,y^4(-6ay^3 + 15x^4),-35x^3y^4\}
			\]
			
			Una vez simplificado en \texttt{Singular}, obtenemos que
			\[
				\{-7 x^6 + 3a x^2 y^3, 3a x^3 y^2 + 5y^4, -y^5\}
			\]
			
			La comprobación de \texttt{Singular} nos dice la base estándar correcta es:
			\[
				\{7 x^6 - 3a x^2 y^3, 3a x^3 y^2 + 5y^4, y^5\}
			\]
			luego, nuestra solución es correcta (salvo constante).
			
			\medskip			
			
			\item \(a \neq 0, b \neq 0\)
			
			En este caso, el resultado que nos arroja \texttt{Python} es el siguiente:
			\[
			\{ax^3y^3 + bx^4y^4 - x^7 + y^5,3ax^2y^3 + 4bx^3y^4 - 7x^6,3ax^3y^2 + 4bx^4y^3 + 5y^4,
			\]
			\[
			ax^3y^3 + 3x^7 + 4y^5,y^3(7ax^3 + 13y^2),y^5(-63a^2x^2y + 336by^4 - 237x^3),y^2(-12a^2x^2y^3 - 8ax^6 + 64by^6 - 60x^3y^2),
			\]
			\[
			y^5(21a + 273bxy),-21a^2x^2y^3 + 49ax^6 + 52by^6,-21a^2x^3y^2 - 35ay^4 + 52bxy^5,y^5(-105ay^3 + 273x^4),
			\]
			\[
			y^6(-230459985a^2y - 1112792499x),y^4(-1857492a^3x^2y + 7930944aby^4 - 7223580ax^3 + 2070432bx^4y),
			\]
			\[
			x^2y^2(-52920ay^3 + 137592x^4),y^4(132300ay^3 - 343980x^4),147a^2x^2y^5 - 3549by^8,
			\]
			\[
			-1323a^3xy^5 - 91728b^2y^9 + 64701bx^3y^5, xy^5(105019740a^3y + 849493008ax + 4451169996bx^2y),
			\]
			\[
			xy^5(-420078960a^3 + 26368956432bx^2),y^5(26460by^4 + 5292x^3),y^4(2160900by^4 + 432180x^3),
			\]
			\[
			4096y^6,2.61310029691576e+16axy^5,536870912ay^5,461569425408x^2y^5,
			\]
			\[
			xy^2(4445280a^2y^3 - 112687848ax^4 - 187813080xy^2),-4226826240y^5,ay^4(8589934592a^2x^2 - 68719476736by^3),
			\]
			\[
			x^2y(9.46987077866619e+16ay^3 - 2.20963651502211e+17x^4),			
			\]
			\[
			-1770209280x^3y^4,5310627840ax^2y^3 - 12391464960x^6\}
			\]
			
			Una vez simplificado en \texttt{Singular}, obtenemos que
			\[
				\{7ax^3y^3+13y^5,49ax^6+52by^6-21a^2x^2y^3, 536870912ay^5,
			\]
			\[
				-112687848ax^5y^2 - 187813080x^2y^4 + 4445280a^2xy^5\}
			\]
			
			La comprobación de \texttt{Singular} nos dice la base estándar correcta es:
			\[
				\{y^5, 49ax^6+52by^6-21a^2x^2y^3, 
			\]
			\[
				180075 a^7 x^3y^2 + 19307236 b y^5 + 300125a^6y^4 \}
			\]
			En este caso, observamos que no obtenemos la misma solución. Esto se debe a que nuestro programa en \texttt{Python} genera una base muy grande (17 elementos) y con unos coeficientes no del todo precisos (los \(10^x\) han perdido decimales). Es por ello que la base que nos arroja tiene 4 generadores, pero puesto que la correcta tiene 3, podemos deducir que la de nuestro programa también lo es pero con falta de precisión.
	\end{itemize}
	\item Representa en \(\mathbb{N}^2\) el ideal líder asociado a cada base estándar.
	
	Para ello cogeremos cada una de las bases obtenidas en cada caso y calcularemos el ideal líder. Es decir, calcular el ideal formado por los monomios líderes de cada elemento de la base estándar y posteriormente representarlos en \(\mathbb{N}^2\). La representación hará referencia a los exponentes de las variables x e y respectivamente. Por ejemplo, en caso de que el monomio líder de un elemento sea $\{2x^2 y^4\}$, la representación de este monomio en \(\mathbb{N}^2\) sería (2,4).
	
	\begin{itemize}
		\item \(a = 0, b = 0\) \\
		En este caso la base que obtuvimos es la siguiente:
		\[
		\{-7x^6, 5y^4\}
		\]
		
		El ideal líder formado por los monomios líderes de cada elemento sería:
		\[
		\{-7x^6, 5y^4\}
		\]
		Expresado en \(\mathbb{N}^2\) obtendríamos:
		\[
		\{ (6,0), (0,4)\}
		\]
		
		
		\item \(a = 0, b \neq 0\) \\
		En este caso la base que obtuvimos es la siguiente:
		\[
		\{13 y^5, -455 x^2 y^4, -1365 x^6, 4b x^4 y^3 + 5 y^4\}
		\]
		
		El ideal líder formado por los monomios líderes de cada elemento sería:
		\[
		\{13 y^5, -455 x^2 y^4, -1365 x^6, 4b x^4 y^3\}
		\]
		Donde podemos observar, que le único elemento de la base es el que realmente cambia. Expresado en \(\mathbb{N}^2\) obtendríamos:
		\[
		\{ (0,5), (2,4), (6,0), (4,3)\}
		\]
		
		\item \(a \neq 0, b = 0\) \\
		En este caso la base que obtuvimos es la siguiente:
		\[
		\{7 x^6 - 3a x^2 y^3, 3a x^3 y^2 + 5y^4, y^5\}
		\]
		
		El ideal líder formado por los monomios líderes de cada elemento sería:
		\[
		\{7 x^6 , 3a x^3 y^2, y^5\}
		\]
		Expresado en \(\mathbb{N}^2\) obtendríamos:
		\[
		\{ (6,0), (3,2), (0,5)\}
		\]
		
		
		\item \(a \neq 0, b \neq 0\) \\
		En este caso la base que obtuvimos es la siguiente:
		\[
		\{7ax^3y^3+13y^5,49ax^6+52by^6-21a^2x^2y^3, 536870912ay^5,
		\]
		\[
		-112687848ax^5y^2 - 187813080x^2y^4 + 4445280a^2xy^5\}
		\]
		
		El ideal líder formado por los monomios líderes de cada elemento sería:
		\[
		\{7ax^3y^3,49ax^6, 536870912ay^5,-112687848ax^5y^2 \}
		\]
		Expresado en \(\mathbb{N}^2\) obtendríamos:
		\[
		\{ (3,3), (6,0) , (0,5), (5,2)\}
		\]
	\end{itemize}
	
	\item Haz un resumen de las páginas 121 a la 130 de ” A singular introduction to commutative algebra”(Referencia principal de la asignatura)
	
	Este apartado se adjunta en un PDF a continuación, por facilidad a la hora de la división de trabajo del grupo.
	\end{enumerate}

\end{document}
